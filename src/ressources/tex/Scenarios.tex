\chapter{Scénarios}
    \section{InitialiserJeu}
        Pour initialiser le jeu, il nous faudra initialiser:
        \begin{itemize}
            \item Les personnages non-déplacables
            \item Le joueur qui peut être soit un humain soit une IA
            \item Le monde
        \end{itemize}
        \subsection{InitialiserJoueursNonDeplacables}
            Pour chaque Personnage Non Deplacable, on va initialiser leur différents attributs:
            \begin{itemize}
                \item Leur \textbf{nom} a une valeur donnée.
                \item Leur \textbf{position} a NULL.
                \item Leur \textbf{carte} a NULL.
            \end{itemize}
            Leur position et leur carte seront redéfinis par la suite, lorsque le monde aura été crée.\\
        \subsection{InitialiserJoueur}
            Le joueur pourra être de 2 types diff\'erents:
            \begin{itemize}
                \item Un joueur humain.
                \item Une IA.
            \end{itemize}
            Puis nous initialiserons notre heros de la même manière que les Personnage Non Deplacables, avec une specification de sa m\'ethode de jouer en fonction de son type (IA ou humain).
        \subsection{InitialiserMonde}
            Pour chaque carte du monde, nous initialiserons cette carte, puis nous placerons les joueurs.
            \subsubsection{InitialiserCarte}
                Pour initialiser la carte, on récupèrera la taille (largeur et hauteur).\\
                Ensuite, on initialisera chaque cellule de cette carte d'une des trois manières possibles:
                \begin{itemize}
                    \item Cellule \textbf{transport}.
                    \item Cellule \textbf{obstacle}.
                    \item Cellule \textbf{accessible}.
                \end{itemize}
            \subsubsection{PlacerJoueurs}
                Pour chaque joueur déjà initialisé, on le placera a sa position dans sa carte, en vérifiant que le placement est valide.\\
                De plus, on mettra a jour ses attributs position et carte.
                \subsubsection{PlacementValide}
                    Pour le placement du joueur, on vérifiera:
                    \begin{itemize}
                        \item Si la nouvelle position est dans la carte.
                        \item Si la nouvelle cellule est accessible, et ne contient pas déjà un personnage.
                    \end{itemize}
    \section{Jouer}
        On \textbf{affiche la carte}, puis on propose au \textbf{joueur de se déplacer}.\\
        On \textbf{lancera une action} si besoin est.
        \subsection{AfficherCarte}
            Pour afficher la carte, on commence par récuperer la carte du Héros.\\
            Une fois récupérée, chaque cellule sera affichée.
        \subsection{Déplacer}
            Pour déplacer le héros, on demandera tout d'abord au joueur de \textbf{choisir son déplacement}, puis on vérifiera que \textbf{ce déplacement est valide}.\\
            Une fois cette vérification faite, on appliquera le déplacement.
            \subsubsection{SaisieDéplacement}
                Le joueur humain rentrera les coordonnées du déplacement souhaité.
	    Le joueur IA choisira les coordonnées de son déplacement. 
            \subsubsection{PlacementValide}
                Ce scénario reste le même que celui précedemment detaillé.
        \subsection{LancerAction}
            Si la cellule dans laquelle se trouve maintenant le héros est associée à une action, alors cette action est effectuée.
    \section{Combat}
        Pour jouer un combat, on commence par \textbf{initialiser le combat}, pour ensuite \textbf{jouer des coups} tant que le jeu n'est pas terminé.
        \subsection{InitialiserCombat}
            Pour initialiser le combat, qui est dans notre cas une Bataille Navale, il faut:
            \begin{itemize}
                \item Initialiser le plateau.
                \item Placer les bateaux.
            \end{itemize}
            \subsubsection{InitialiserGrilles}
                Pour chacun des deux joueurs en lice, on va initialiser sa grille. On r\'ecupèrera donc la taille de sa grille, pour ensuite créer toutes les cases.
            \subsubsection{PlacerBateaux}
                Une fois les grilles initialisées, chacun des joueurs placera ses bateaux.\\
                Dans le cas des Personnage Non Deplacable, cela fait automatiquement, alors que dans le cas du héros une saisie sera demandée.
                \subsubsection{ChangerCarte}
                    Si le héros se trouve sur une cellule transport, il est d\'eplacé sur la carte correspondante, aux coordonnées corespondantes.
                \subsubsection{MettreAJourComp\'etences}
                    À l'issue d'un combat, le joueur victorieux r\'ecupère des comp\'etences suppl\'ementaires.
                \subsubsection{JouerCombat}
                    Détails ci-dessous.
        \subsection{JouerCombat}
            À chaque tour de jeu, on \textbf{affiche les deux grilles}, puis le joueur courant \textbf{joue son coup} tant que la \textbf{partie n'est pas finie}.
            \subsubsection{AfficherCombat}
                On affiche deux grilles:
                \begin{itemize}
                    \item La grille du joueur courant (compl\'etement dévoilée)
                    \item La grille du joueur adverse (dont seule les cases déjà visées seront montrées)
                \end{itemize}
            \subsubsection{JouerCoup}
                Jouer un coup consiste en:
                \begin{itemize}
                    \item Une choix de coup
                    \item Une vérification de la validité du coup
                    \item Le placement du coup
                    \item Le changement du joueur courant
                \end{itemize}
                    \subsubsection{SaisirCoup}
                        Le joueur courant humain saisie les coordonnées du tir tandis que l'IA calcule les coordonnées de son coup. 
                    \subsection{CoupValide}
                        On vérifie que:
                        \begin{itemize}
                            \item les coordonnées appartiennent bien à la grille.
                            \item La case n'a pas déjà été visée.
                        \end{itemize}
                    \subsection{PlacerCoup}
                        On effectue les changements associés à ce coup sur la grille:
                        \begin{itemize}
                            \item Rév\'elation de la case visée.
                            \item Si cette case contient un bateau auquel retire un point de vie.
                        \end{itemize}
                    \subsection{ChangerJoueur}
                        Le joueur courant devient le joueur adverse et vice-versa.
            \subsubsection{PartieFinie}
                On vérifie si tous les bateaux d'un des joueurs ont été coulés.
