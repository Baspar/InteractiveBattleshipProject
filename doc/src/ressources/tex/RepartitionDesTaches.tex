\chapter{Répartition des tâches}
    Pour ce faire nous avons donc décidé d'oganiser des rendez-vous en groupe complet 
    afin de générer le plus d'idées et d'avis possibles. Une fois la concertation 
    terminée nous obtenons un plan de travail. C'est ainsi que nous avons réalisé notre 
    diagramme uml (grâce au logiciel \textbf{PlantUML}), les diagrammes de séquence ainsi que les objectifs de notre projet.\\

    Une fois l'étape de pré-conception réalisée, nous nous sommes donc attaqués à l'étape de 
    programmation. Notre méthode a été lors de cette étape de se répartir le travail 
    selon l'affinité de chacun vis-à-vis des différents compartiments du projet. Nous 
    avons ainsi utilisé l'outil \textbf{Git} afin de permettre à tous d'avoir accès à un dossier 
    commun ce qui permet alors une grande souplesse dans l'organisation du travail de 
    chacun. Le travail de chacun consistait alors de d'abord commenter chaque classe 
    afin de déterminer les différentes étapes de conception et ainsi d'avoir une vision 
    globale du travail à réaliser. Le commentaire des classes a été fait à l'aide de 
    l'outil \textbf{Doxygen}.\\

    \renewcommand{\labelitemi}{$\rightarrow$}
    Le déroulement du travail de chacun se décomposait alors comme suit:
    \begin{itemize}
        \item Création des classes selon le modèle prédéfini
        \item Commentaire des classes
        \item Création du code
        \item Réalisation des tests unitaires\\
    \end{itemize}

    Chacun veillant alors à travailler sur une part propre du projet afin de ne pas 
    provoquer de collisions avec les travaux un autre membre du groupe. Il était cependant possible de 
    travailler sur la même partie du projet grâce au logiciel \textbf{Git}. Les collisions étant 
    directement gérées par le logiciel.\\
