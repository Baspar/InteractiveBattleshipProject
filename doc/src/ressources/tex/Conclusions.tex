\chapter{Conclusion}
    \section{Synthèse critique}
        Dans l'ensemble, l'avis général de chacun des membres du groupe vis-à-vis des méthodes de travail est positif. En effet, dûment à la taille conséquente de notre groupe, la quantité de travail demandée au niveau de la réalisation du projet était conséquente, ce qui nous a fait considérer cette méthode de travail comme la plus adaptée. Elle n'était pas forcément optimale mais elle était la plus simple à réaliser et à mettre en place afin d'être opérationnelle le plus rapidement possible. Ceci a notamment était possible grâce à la coopération de chacun. Les parties les plus techniques étant ainsi déléguées à nos plus valeureux programmateurs afin que chacun puisse contribuer au projet à hauteur de ses compétences. Nous avons ainsi pu tous progresser dans notre manière d'aborder un grand projet et également dans la manière de programmer en C++.\\ 
    \section{Conclusions personnelles}
        \subSection{Kevin Nguyen}
            De grosses appréhension face à la quantité de travail demandée pour réaliser ce projet. Cependant, une organisation efficace ainsi qu'un groupe solidaire nous ont permis de mener à bien cette mission. Un projet qui en soi nous a permis de progresser chacun à notre rythme sans ressentir une quelconque gêne vis-à-vis de la quantité de travail effective réalisé par chacun (les capacités de chacun étant valorisées en fonction de notre domaine de compétence).
        \subSection{Thibaut Chavane}
            Ce travail a été intéressant, tant par le travail qu'il représentait en lui-même (coder un jeu avec de nombreuses possibilités, et deux grandes sous-parties quasiment indépendantes), mais aussi par les méthodes auxquelles nous avons eu recours au cours du semestre, comme par exemple l'utilisation de Git, Doxygen, PlantUML. \\
            De plus, le nombre conséquent de membres du groupe nous a permis de nous entraîner à la gestion d'une équipe, plus que lorsqu'on effectue des projets en groupes plus réduits (de deux ou trois étudiants par exemple).
        \subSection{Clémentine Cavaroc}
            Ce projet nous a à tous paru ambitieux car il représentait une masse de travail considérable, mais grâce à une répartition des tâches efficace et une bonne gestion du temps, nous sommes parvenus à mener à bien ce projet dans les temps. \\
            L'avantage d'avoir réparti les tâches en fonction des niveaux en programmation de chacun est que chacun a pu progresser et contribuer à l'avancement global du projet en même temps, ce qui est après tout le but de chaque projet.
        \subSection{Léo Lhuissier}
            L'idée de créer un jeu imbriqué dans une autre a été très instructif car c'est quelque chose que nous n'avions pas eu l'occasion d'expérimenter auparavant. \\
            De plus, le fait d'avoir été un groupe plutôt nombreux nous a permis de nous entraîner à la gestion d'une équipe, ce qui fait partie intégrante du métier d'ingénieur.
        \subSection{Bastien Laine}
            Lors de ce projet, j'ai eu l'occasion de découvrir de nouvelle technologies, ainsi que de continuer mon apprentisage sur d'autre. \\
            De plus, une chose non quantitative que m'aura appris ce projet est le travail en équipe en plus grand nombre. En effet jusqu'ici, aucune matière ne nous à poussé à travailler à 7, et ce sur une tâche d'une empleur aussi conséquente. Cela nous a permis de mettre en place de nouvelle méthode de travail, de division de travail, ou encore de communication, qui jusque là n'avaient pas été necessaire. \\
            Il est clair que le projet aurait pu être amélioré, nottament sur une plus grande "généralité" du code ( Par exemple via des classe IHMCombat, ControleurCombat, ect...). \\
            Cependant à notre niveau, le travail est satisfaisant, surtout de par le fait que nous avons pu implémenter de nouvelle Arme, ou IA de manière très rapide, montrant l'efficacité de la généralite de notre code.
        \subSection{Anne Besnard}
            Grâce à ce projet, j'ai eu la chance d'experimenter le travaille en groupe. En effet, nous travaillions à sept, ce qui a demandé une organisation plus évolué que pour les projets en binômes et a permis d'apprendre des autres sur le plan programmation autant que sur l'informatique en général\\
            Pour faciliter la coordination des membres du groupe,  nous avons utilisé git. Lors de mon stage technicien, j'ai eu l'occasion de découvrir git et son fonctionnement, ce projet m'a permis de voir une utilisation concrète et utile de ce logiciel.\\
            Finalement,ce projet a été l'occasion d'améliorer mes compétences en algorithmie, en programmation orientée objet et bien sûr d'apprendre la syntaxe du C++ que je ne connaissais pas. 
