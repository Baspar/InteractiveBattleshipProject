\chapter{Conclusions personnelles}
    \section{Thibaut Chavane}
        Ce travail a été intéressant, tant par le travail qu'il représentait en lui-même (coder un jeu avec de nombreuses possibilités, et deux grandes sous-parties quasiment indépendantes), mais aussi par les méthodes auxquelles nous avons eu recours au cours du semestre, comme par exemple l'utilisation de Git, Doxygen, PlantUML. \\
        De plus, le nombre conséquent de membres du groupe nous a permis de nous entraîner à la gestion d'une équipe, plus que lorsqu'on effectue des projets en groupes plus réduits (de deux ou trois étudiants par exemple).
    \section{Clémentine Cavaroc}
        Ce projet nous a à tous paru ambitieux car il représentait une masse de travail considérable, mais grâce à une répartition des tâches efficace et une bonne gestion du temps, nous sommes parvenus à mener à bien ce projet dans les temps. \\
        L'avantage d'avoir réparti les tâches en fonction des niveaux en programmation de chacun est que chacun a pu progresser et contribuer à l'avancement global du projet en même temps, ce qui est après tout le but de chaque projet.
    \section{Léo Lhuissier}
        L'idée de créer un jeu imbriqué dans une autre a été très instructif car c'est quelque chose que nous n'avions pas eu l'occasion d'expérimenter auparavant. \\
        De plus, le fait d'avoir été un groupe plutôt nombreux nous a permis de nous entraîner à la gestion d'une équipe, ce qui fait partie intégrante du métier d'ingénieur.
    \section{Bastien Laine}
        Lors de ce projet, j'ai eu l'occasion de découvrir de nouvelle technologies, ainsi que de continuer mon apprentisage sur d'autre. \\
        De plus, une chose non quantitative que m'aura appris ce projet est le travail en équipe en plus grand nombre. En effet jusqu'ici, aucune matière ne nous à poussé à travailler à 7, et ce sur une tâche d'une empleur aussi conséquente. Cela nous a permis de mettre en place de nouvelle méthode de travail, de division de travail, ou encore de communication, qui jusque là n'avaient pas été necessaire. \\
        Il est clair que le projet aurait pu être amélioré, nottament sur une plus grande "généralité" du code ( Par exemple via des classe IHMCombat, ControleurCombat, ect...). \\
        Cependant à notre niveau, le travail est satisfaisant, surtout de par le fait que nous avons pu implémenter de nouvelle Arme, ou IA de manière très rapide, montrant l'efficacité de la généralite de notre code.
